% hello world example
% compile with: xelatex example.tex

% you can also use 'book' if desired
\documentclass{article}

% lets you use the href command
\usepackage{hyperref}

\begin{document}

This document is an \emph{example/notes} for learning \LaTeX (using \href{https://www.tug.org/twg/mactex/tutorials/ltxprimer-1.0.pdf}{this book} ).
Continuing the text on this line (rather than above) makes no difference for the paragraph.

The blank line above forces a new paragraph to start here.
Having multiple         spaces between words doesn't make a difference (it's treated as one space).
This symbol \ forces an extra space to be added!
Use two backslashes to force\\
a line breakpoint.

testing other languages: aquí, γεια σας (this doesn't work directly, at least without some greek font).


\noindent This paragraph will start without an indent!
% insert content from another file:
this text comes from a separate file!


\begin{center}
  \section{Spacing:}
\end{center}

\begin{flushright}
  This is aligned to the right.
\end{flushright}
\begin{flushleft}
  This is aligned to the left.
\end{flushleft}


\section{Bullets}
Use the \underline{itemize} \href{https://www.overleaf.com/learn/latex/Environments}{environment} with the \underline{item} command (see fonts section below for an example).


\begin{center}
  Fonts:
\end{center}


\begin{itemize}
  \item A (font) type style is defined by \emph{family > series > shape}.
  \begin{itemize}
    \item \textsf{\textbf{this is: sans serif > boldface > upright (default)}}
    \item \textrm{\textmd{\textit{this is: roman > medium > italic}}}
    \item \textrm{\textmd{\textsl{this is: roman > medium > slanted}}}
  \end{itemize}

  \item The emph command is just a shorthand that will use the \emph{italic} or \emph{upright} shape as needed to be the opposite of it's surrounding text.
  \item We \tiny{can} also \scriptsize{play around} with \huge{font size}.  normalsize will \normalsize{} change things back to normal.
\end{itemize}


\section{
  \href{https://libguides.utsa.edu/c.php?g=522165&p=3570198}{The Document}
}
The document class uses the syntax documentclass[options]{class}, for example documentclass[11pt]{article}

...



\end{document}
